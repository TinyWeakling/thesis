% !Mode:: "TeX:UTF-8"

\chapter{绪论}\label{chap:introduction}
随着“绿色科技”概念的提出,能源节约,环境保护和绿色能源应用等概念渐渐深入人心。
绿色科技将会成为下一轮的工业革命的重点内容,所以建设高效,节约并环保的绿色科技产业对于科技的进步,国家的可持续发展都具有重要的意义。
以发光二极管(Light Emitting Diodes, LEDs)代替传统的白炽灯和荧光灯的系列措施,就是绿色科技在实践生活中的一个重要体现。

\section{论文研究背景及意义}\label{sec:background}
\subsection{研究背景}
有着“绿色固体光源”之称的LED在照明领域的发展越来越迅速,资料表明,白光发光二极管(White Light Emitting Diodes, WLEDs)在能效表现上,是传统光源能效的20倍,是荧光灯的5倍。
若所有家庭彻底更新至固态照明将使得全球用电量降低50\%\cite{DingJuPeng2005}。此外,LED照明技术还具有节能,使用时间长,可控性好,绿色环保等诸多优点。
同时,由于近年来无线通信技术的不断发展,频谱短缺问题也越来越严重,利用常见频段进行高速无线通信的成本也极大地提高,这迫使人们采用可利用的其他频段去解决频谱资源问题。
而半导体LED器件,也由于其响应灵敏度高,易调制等优越性能,可以将其广泛地用于无线通信领域,解决上述的频谱短缺现象\cite{KomineT2004}。在这些因素的不断推动下,可见光通信技术也就应运而生。

可见光通信作为一种新的宽带无线接入技术,充分结合了LED照明技术和无线通信技术的优势。
利用在普通家庭中被大规模应用的半导体LED照明设施,可以提供了一种全新的接入方式,满足人们的高速通信需求。
在可见光通信系统中,LED具有照明和通信的双重功能\cite{SongZhengXun2008},一方面,由于白色LED的调制速率非常高,在一定的范围内可以使人眼完全感受不到,同时也不会对人眼产生危害,
因此使用白光LED进行通信并不会破坏原有的LED照明属性。另一方面,由于白光所处的频带较高,LED设备可以代替无线局域网中的传统的接入点设备,并获得更高的传输速率。
且由于光传播不会具有无线电波的危害,因此还可以用于很多无线电波不能应用到的场合,例如医院,飞机舱等\cite{ChenRan2013}。
因此,可见光通信将是一种频谱资源丰富,通信容量大,绿色安全,保密性能好,且无需许可证的高效短距离通信方式。

随着无线通信技术的不断发展,人们对通信速率的要求也不断提高,而室内短距离无线通信技术作为人们最常使用到的通信技术,
也逐渐从传统的红外线,紫外线通信,到蓝牙通信,再到无线局域网通信进行过渡。光通信技术的发展将是室内短距离无线通信发展的一个重要机遇。
它可以为人们提供一个高速且绿色环保的通信体验。

\subsection{研究意义}
可见光通信作为一种新兴的无线光通信技术,随着研究的不断深入和LED技术的不断发展,将会获得极大的应用空间。
目前,已知的应用领域有照明与通信,视觉信号与数据传输,显示与数据通信和室内定位等\cite{XueXiaoMan2014}。
随着LED在室内照明领域的不断被使用,可见光通信也将会得到迅速的发展。

目前针对于可见光通信的研究主要集中在灯组元件的设计,物理层点对点通信这个阶段,对于上层室内组网的系统设计的研究仍然较少\cite{HuangZT2012},
而要想实现室内可见光通信系统的实际应用,必须依靠多个灯组的有效组织和管理,因此随着可见光通信的进一步发展,组网技术必将会成为光通信的一个重要组成部分。
在目前物理层研究已经基本成熟的阶段,进行光通信的上层组网技术研究是一个绝佳的时机,为光通信在实际生活中的应用奠定了良好的理论基础。

\section{国内外研究现状}\label{sec:stage}
\subsection{国外研究现状}
由于可见光通信的广泛的应用前景和重要的实用价值,国外的一些研究机构和运营商在很早之前就开始了可见光通信的研究。
特别是光通信概念的提出国家日本,技术发展的引领者美国,欧洲等在可见光通信的研究中都获得了很有意义的研究进展。

可见光通信最先的理论研究在日本展开。
从2000年开始,日本中川实验室的研究人员就对基于白光LED的可见光通信系统的信道进行了初步的数学分析和仿真建模,论证了使用白光LED作为一种照明和通信光源的可能性\cite{Tanaka2000}。
在2002年,中川实验室的研究人员又对可见光通信系统的光源属性,光源位置,信道模型,室内信噪比分布,符号间干扰影响等内容进行了详细的分析和讨论,为可见光通信的理论分析奠定了一个重要基础\cite{KomineT2004}。
在2003年,日本成立了致力于推动光通信发展的组织可见光通信协会(Visible light communications consortium, VLCC),并和多家日韩企业合作,如索尼,三星,东芝等,共同开始可见光通信的推广和标准化工作\cite{VLCC}。
VLCC在可见光通信领域已经取得了很多的研究进展,如三星公司展出过工作距离为1米的100Mb/s的双向可见光通信系统;中川工作室开发了基于可见光通信的超市定位和导航系统\cite{VLCC}。

欧洲为了抓住光通信发展的机遇,也联合了20多家大学科研单位和企业,建立了OMEGA计划来组织光通信的研究工作\cite{OMEGA}。
可见光通信作为OMEGA计划的一个重要组成部分,被用来提供一种新式的宽带高速服务。该计划的持续推进,使得可见光通信技术有了重要的发展。
2009年,牛津大学的Dominic O'Brien教授等人利用均衡技术实现了100Mb/s的可见光通信技术\cite{le2009},在2010年,他们又把当前LTE系统的关键技术MIMO和OFDM技术应用于可见光通信中,实现了220Mb/s的传输速率\cite{azhar2010}。
同年,来自德国的科研人员利用普通商用的白色LED灯,刷新了可见光通信系统的速率峰值,达到了513Mb/s,并认为系统速率还有提升的空间\cite{vuvcic2010}。
2011年,上述团队又利用三色光型的白色LED和密集波分复用技术,再次刷新了可见光通信系统的速率极限,获得了803Mb/s的峰值速率\cite{vucic2011}。

美国也是促进可见光通信技术发展的一个重要的国家,其依托于加州大学和国家实验室的UC-light机构是研究可见光通信的重要国际研究中心。该中心在LED照明,光学器件研究,光通信组网等方面都有着详实的研究成果\cite{UcLight}。

国际上对于可见光通信系统的研究,除了着眼于上述的物理层通信速率的提升的层面上,在光通信组网领域,也有着一些前期研究成果。

对于灯组布局优化问题,文献\cite{KomineT2004}中,在对可见光通信的信道模型进行了详细分析的基础上,提出了一种在典型场景下的灯组布局,并仿真了在该灯组布局下的光照分布和接收端接收功率和信噪比的情况,
仿真结果表明该灯组布局可以满足一般照明需求,并获得较好的信息传输速率。但是论文中仅仅提出了经典的房间模型,并使用了典型的灯组布局参数进行仿真,并没有对灯组布局进行优化。
文献\cite{AzizanLA2012}中为了使的接收端信噪比的分布更加均匀,采用了另外一种灯组布局方式,在房间的正中心安放一个由多个LED组成的灯组,在房间的四个角落分别安放由较少LED组成的灯组。
这种方式虽然对系统接收端信号的均匀度有一定的帮助,但是其实用性还有待检验。文献\cite{DingDeQiang2007}在文献\cite{KomineT2004}灯组布局模型的基础上,分析了灯组设计与接收光功率分布的关系,
以室内光接收功率的均匀化作为目标,给出了在四个通信光源的房间内的最优化灯组布局设计。文献\cite{DingJP20121}\cite{DingJP20122}中则同样为了使得灯组的光接收功率和光照前度达到均匀,采用了基于遗传算法的最优化模型对系统进行了建模,
通过灯组的发射功率强度系数进行系统优化,并对16个LED灯组和25个LED灯组的情况进行了仿真。

对于灯组的调度和协作问题,文献\cite{Vegni2012}对灯组进行了分类,分为均匀灯光的灯组和点光源的灯组,对于上述两种情况,分别研究了在两个灯组有重叠和有间隔情况下,如何进行协调和协作,从而保持连接的平滑过渡。
该方案只提及到两个灯组之间的协作问题,并没有对多灯组组网的情况进行讨论。文献\cite{LiYY2012}对光通信的灯组进行了两点假设,即通信时灯组的辐射角度可以进行调节且灯组的传输方向可以进行调节,
并通过仿真证明上述的两个改变不会影响人眼的观察。在这基础上,设计了光通信组网的一个调度框架,来提高系统的吞吐量。文献\cite{HuangZT2012}描述了在一个屋子内有很多的灯组的场景,这些灯组都被连接在一个LED调度器上,
LED调度器负责配置相应的灯组和时间资源来和用户进行数据传输,并提出了一种基于带宽的静态用户灯组选择算法来解决多用户的同时通信问题。

\subsection{国内研究现状}
国内对可见光通信的研究起步较晚,目前仍处于追赶的状态,进行研究的团体主要为国内的著名高校和有实力的公司,如清华大学,东南大学,浙江大学,暨南大学,西安理工大学等,研究的方向也正由局部性的问题探讨到通信系统整体性地硬件设计,对此本文进行了一些总结。

2006年,暨南大学的陈长缨教授设计了基于LED照明通信的点对点通信系统,并实现了带宽为10MHz条件下的数据传输,验证了该通信系统的通信性能\cite{hu2006}。
同年,北京大学在进行了系列理论研究和仿真的基础上,提出了基于广角镜头的超宽视角的可见光通信方案,并进行了硬件实现,
在2010年的英特尔杯大学生嵌入式系统竞赛参赛作品中,实现了拥有5个广播频道,传输距离6米,速率为3Mb/s的可见光通信系统\cite{ChenTe2013}。

2008年,陈长缨再次设计出基于白色LED的可见光通信系统模型,该通信系统的传输距离超过2.5米。
同年,西安理工大学的研究人员采用非线性回归的轨迹算法,对商用的LED发光模型进行了研究和数学分析,并设计和优化了白光LED的阵列\cite{Ding2010}。

2013年4月,在河南郑州,国家863计划"可见光通信系统关键技术研究"主题项目正式启动,该项目由解放军信息工程大学联合国内多家优势单位共同承担,
旨在开发可见光频谱资源,研究可见光通信系统在复杂信道条件下的关键技术,建立可见光通信实验系统并开展典型应用示范,为可见光通信这一新型绿色信息技术的产业化奠定基础\cite{VlcProject}。

2013年10月,复旦大学计算机科学技术学院宣布成功实现利用室内可见光来传播网络信号。研究人员将一盏1W的LED等接入因特网,
灯下放置的4台电脑即可进行上网,最高单向传输速率达到3.7Gb/s,刷新了可见光领域的单向传输速率记录,同时平均速率可达150M\cite{Lifi}。

我国目前正在制定LED市场化的战略,LED代替传统的照明设备必定是不可阻挡的趋势,而将照明和通信结合起来的可见光通信技术也将会成为一个高效的技术手段,使得室内高速宽带通信变成可能。


\section{论文主要研究工作和章节安排}\label{sec:concept}
本论文主要进行了如下工作。首先,对LED的发光原理,物理特性,以及可见光通信系统信道模型进行了介绍和理论分析,接着对常见室内灯组分布场景下的LED通信系统室内光照强度和光接收功率的分布进行了研究,
并采用优化算法,搜索出在给定场景下LED灯组的最优光源布局方式。同时,针对于不同的灯组组网方式,分别对分布式灯组组网和独立式灯组组网下多个移动用户的场景进行了分析,
并采用了有效的灯组调度方式来解决多用户通信问题,保证通信中连接的持续性和可靠性。

本论文一共分为五个章节:

第一章主要介绍了可见光通信系统的背景和国内外的发展概况,并着重介绍了可见光通信在组网技术方面的研究成果,同时阐述了所研究的光通信组网技术对于科技进步和社会发展的重要意义,并介绍了论文研究内容的主要工作重心。

第二章主要介绍了室内可见光通信系统的理论基础,如LED的发光特性,光通信的链接方式,光通信的信道模型等基本内容,从而对后续章节的理论分析奠定理论依据。
另外,还介绍了目前常见的通信系统中使用的组网技术,这里分别以蜂窝移动通信系统和短距离无线局域网为例进行了讨论。
在对常用系统中的组网技术进行研究之后,还对本文的可见光通信系统在组网方面可能会使用到的关键技术进行了分析,从而明确本文所需要解决的技术问题和常见的技术手段。

第三章主要研究了室内灯组的最优化布局方式。该章首先介绍了LED灯的发光原理,经典的室内灯组组网模型和灯组的具体参数。
对于上述灯组组网模型,建立了一个多目标的最优化模型,来协调灯组的布局对光接收功率和室内的光照强度的影响。
对于上述多目标最优化模型,采用了线性加权法对模型进行了简化,并通过计算搜索出最优的布局方式。

第四章主要是对室内可见光通信分布式灯组组网下的灯组调度算法进行了研究。
分布式的组网方式是指所有的灯组只是单纯提供数据的传输作用,而整个网络的控制功能都是由和灯组相连的调度器负责进行的,在这种网络下,用户在灯组之前的移动不需要进行切换操作。
基于该网络架构,提出了一种基于用户运动方向的灯组协同调度算法来解决用户运动过程中在灯组的边缘区误码率过高的问题,在灯组的边缘覆盖区利用灯组之间的协作来进行数据传输。
相对于原有的灯组调度方案,该算法利用了用户在运动过程中的方向信息,从而可以更加准确地设置灯组的调度方案。

第五章则主要对室内可见光通信独立式灯组组网下的灯组调度算法进行了研究。
独立式的组网方式则是对于每一个灯组而言,都具有用户接入认证,网络管理等控制功能,各灯组都是功能齐全的独立的通信个体。
对于这种网络结构,提出了一种增量式的灯组调度算法来进行灯组的动态调度。增量式的调度方式是一种基于时分复用的调度框架,由全局调度算法和局部调度算法组成。
在系统的调度过程中,使用一次全局调度根据当前所有用户的情况,选择合适的灯组为多用户进行服务,并在全局调度周期内,使用多次局部调度,
根据移动用户的情况,对之前的调度结果进行合理的调整,从而在减少整体系统的运算量的情况下,保证系统的系统容量。

第六章总结了全文工作,并提出了可见光通信在组网技术方面在本文研究的基础之上可以继续研究的方向和追求的目标。 