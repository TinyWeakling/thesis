% !Mode:: "TeX:UTF-8"

\begin{Resume}
    \begin{itemize}
        \item 期刊论文
            \begin{itemize}
                \item 戴咏玉,潘乐园,宋扬,“尖劈吸波体与导引仿真用微波暗室的性能研究”,《数学的实践与认识》,第42卷第14期,第204--213页,2012年7月.
                \item Y. Dai, S. Jin, L. Pan, et al., ``Interference control based on beamforming coordination for heterogeneous network with RRH deployment'', submitted to IEEE System Journal, Sept. 2012, revised.
            \end{itemize}
        \item 会议论文
            \begin{itemize}
                \item L. Pan, Y. Dai, H. Zhang, et al., ``A novel positioning scheme of unknown radio interference source in cellular networks'', IEEE Wireless Communications and Signal Processing, Oct. 2012.
                \item L. Pan, ``一种改进的LTE下行链路信道估计算法'', 第26届南京地区研究生通信年会,2011年11月.
                \item Y. Dai, S. Jin, L. Pan, et al., ``Adaptive mode switching based on statistical CSI for downlink MIMO in heterogeneous network with RRH deployment'', IEEE Vehicular Technology Conference 2013 (Spring), Jan. 2013.
            \end{itemize}
        \item 专利
            \begin{itemize}
                \item 张华,潘乐园,卞青,赵嘏,“一种应用于3GPP LTE系统的自适应信道估计方法”,申请号:201110427542.8,2011年12月.
            \end{itemize}
%        \item 所获奖励
%            \begin{itemize}
%                \item 东南大学科研创新单项奖学金,2012年11月
%                \item 东南大学优秀研究生干部,2012年11月
%                \item 华为奖学金,2012年5月
%                \item 全国研究生数学建模竞赛二等奖,2011年11月
%            \end{itemize}
    \end{itemize}
\end{Resume}
