% !Mode:: "TeX:UTF-8"

% English Abstract
\begin{englishabstract}{OWC\quad{}VLC\quad{}Optical Channel Modeling\quad{}OFDM\quad{}Adaptive Modulation\quad{}Equalization\quad{}OWC Hardware Design}

By recent years, the rapid development of wireless communications has witnessed the significant shortage of the available spectrum resources and the communication cost using current spectrum is substantially increasing.
Moreover, more and more people have diverted their attentions to the impact of electromagnetic radiation on their physical health, especially the radiation hazard caused by indoor wireless equipment.
Optical wireless communications (OWC) based on light emitting diodes (LED) have been one of the focus research area in recent years for plenty of advantages, such as the unlimited amount of spectrum resources, no license required, high ratio of energy transforming, the green communications with security, etc.
With respect to OWC, this dissertation focuses on the research of key technologies of indoor visible light communications (VLC) and an entire hardware system is also designed to demonstrate the availability of VLC systems.

Firstly, this thesis shows the overview of the indoor VLC systems.
The thesis introduces the development history of VLC at first.
Considering the difference between VLC and the conventional radio frequency (RF) communications, we describe the specific characteristics of the VLC information channel.
Furthermore, we introduce the basic features of optoelectronic devices including LED and photodiodes (PD) and present the basic circuits structures while the primary parameters are measured in the designed circuits. At last, various kinds of noise sources in VLC systems are introduced in the thesis.

Additionally, the methods of optical channel modeling are investigated for the light of sight (LOS) propagation channel in indoor VLC systems.
To begin with, we present a modeling method to model the linear optical channel based on the measured value to fulfill the requirements of linear system simulations in the case of that the emitting power is not so large that little nonlinear distortion emerges in the analogue components.
On top of that, we also present a more practical and accurate optical channel model which is the nonlinear model with the observed data measured by the vector network analyzer while the emitting optical power is large enough to draw the nonlinear distortion in the VLC systems.
Lastly, we compare the modeling channels with the measured data and the results confirm that the channel models are both precise enough to be used in numerical simulations.

Furthermore, we investigate the key baseband technologies of VLC system based on the optical channel models set up in the thesis.
Firstly, the spectral efficiency and power efficiency of various baseband modulation schemes are presented, including OOK, PPM, direct-current optical PAM (DCO-PAM) and direct-current optical OFDM (DCO-OFDM).
On top of that, we adopt DCO-OFDM which is more adaptive to wide-band communications as the modulation scheme to evaluate the adaptive modulation technology in VLC systems.
The numerical results show that the spectral efficiency of DCO-OFDM is substantially increasing with SNR under a specific BER performance which performs much better than fixed modulation schemes.
Lastly, we present two kinds of equalizers which are the digital pre-equalizer and the analogue post-equalizer.
The numerical results show that a data rate of 200 Mbps can be achieved by using OOK NRZ modulation with the BER around $2\times10^{-3}$ when the system average SNR is 10 dB.
Additionally, the 3-dB bandwidth of the optical channel is approximately increased to 80 MHz when the first-order analogue post-equalizer is employed in the optical receiver which is the foundation to realize the wide-band single carrier VLC systems.

Finally, we design a real confirmation hardware platform of indoor VLC systems and demonstrate a streaming video broadcasting system which significantly confirms the realizability and availability of indoor VLC systems.
We present the overview of indoor VLC confirmation hardware systems and show design schemes of the optical transmitter and receiver while the Ethernet interface is adopted to connect the VLC systems with the personal computers (PCs) in the thesis.
Additionally, all the hardware modules are minutely measured and evaluated in the thesis.
Consequently, we complete an entire indoor VLC confirmation platform and achieve a data rate of 150 Mbps using OOK-NRZ modulation with which a streaming video broadcasting system is realized and a 3D 1080p high-definition movie has been transmitted through the VLC system under a distance of about 2 meters.

\end{englishabstract}
