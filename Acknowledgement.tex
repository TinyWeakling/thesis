% !Mode:: "TeX:UTF-8"
% 致谢

\begin{Acknowledgement}
时光荏苒,岁月如梭,转眼间两年多的研究生生活就要结束了。
在论文即将完成之际,笔者谨向在硕士研究生学习阶段所以给予我培养教育、帮助支持的老师、同学、朋友和亲人致以最诚挚的谢意!

衷心感谢我的导师赵春明教授自大学三年级以来对我在学习上的悉心指导和生活上的关怀备至。
赵老师拥有广博的学识和严谨的治学风格,他对许多科研问题都有着开阔的思路和精辟的见解,更令人敬佩的是他对科研和教育事业的献身精神和崇高品格,给我以深刻的影响并将成为我毕生做人和治学的楷模。
赵老师为我们提供了良好的科研环境,引导我进入国内外知名的移动通信研究前沿——东南大学移动通信国家重点实验室,他为我们创造了密切合作的学术团队和科研氛围。
在这里我的科研创新能力和团队合作意识得到了极大地锻炼与提高,我的理论知识和实践经验都取得了长足地进步,培养了比较系统的思维方式和工作方法。
在这几年学习生活中,我的每一点进步都蕴含着赵老师的心血。

衷心感谢张华老师对我循循善诱的指导,培养了我遇到困难、解决困难的能力。
张老师教会了我如何从不求甚解到知其然且知其所以然。
是张老师解决问题缜密的思维方式使我知道了什么才是真正的研究。
张老师丰富的学识与严谨的研究作风,是我在今后研究生生活中继续学习的榜样!

特别感谢王家恒老师、许威老师、姜明老师、黄鹤老师、傅学群老师和杜永强老师在LTE项目组和光无线通信项目组中对我的耐心指导和细心帮助,在课题遇到困难时他们总是给予我鼓励和信心,他们丰富的学识和经验引导我找到解决问题的方法和途径。
在这段时间里,我对LTE系统的认识、对研究课题的理解以及对工作学习的态度,都有很大的进步,这些都离不开他们的精心指导。

感谢本实验室金石副教授、尤肖虎教授、沈连丰教授、陈明教授和蒋良成教授以及教务员房芳老师在学习和生活上给予我的关心和帮助。

特别感谢谢鑫、赵慧霞、丁海燕、孟向阳、冯义亮、左大华、陈春艳、吴超培、刘亦辰等同学对我的帮助与支持,没有和他们亲密无间的交流与合作,我无法在如此短暂的时间里取得这些成果。

感谢杨逾山、陆有懿、洪赟、金毅、廖慧兵、沈弘、刘睿、王晨、卞青、赵嘏、孙晓星、赵欢、朱道华、蔡菁菁、朱琳、徐挺玉、来晓泉、汪莹、沈启辰、杨飞、黄健、周煌等师兄弟姐妹的帮助,
与他们在一起的学习交流和共同生活给了我莫大的进步和快乐。
感谢实验室其他众多优秀的老师和同学,从他们身上学到的很多东西,将使我终身受益。

特别感谢我的家人,他们一如既往的支持和鼓励始终伴随着我,使我能够安心学习。

最后,我要特别感谢我的女友戴咏玉,她的理解、支持、鼓励与帮助一直是,也将永远是我奋进的动力。

\begin{flushright}
潘乐园 \\
2012年12月31日\\
\end{flushright}

\end{Acknowledgement} 