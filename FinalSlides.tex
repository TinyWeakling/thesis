% !Mode:: "TeX:UTF-8"
\documentclass[mathserif, utf8, 9pt]{beamer}
\usepackage{eulervm}
% Setup appearance:

\mode<presentation>
{
  %\usetheme{Warsaw}
  %\usetheme{Darmstadt}
  %\usetheme{Berkeley}
  \usetheme{Berlin}
  \usecolortheme{whale}
  \setbeamercovered{transparent}
}
\hypersetup{pdfpagemode=FullScreen}
\synctex=1
\usepackage{xeCJK}
%
\setCJKmainfont[Mapping=tex-text]{SimSun}
\setCJKmonofont[Mapping=tex-text]{SimSun}
\setmainfont[Mapping=tex-text]{Times}
\setmonofont[Mapping=tex-text]{Courier New}
\setsansfont[Mapping=tex-text]{Trebuchet MS}

\setbeamertemplate{caption}[numbered]

% Standard packages
\usepackage{times}
\usepackage[T1]{fontenc}
\usepackage[utf8]{inputenc}
\usepackage[english]{babel}
\usepackage{colortbl}
\usepackage{booktabs}
\usepackage{longtable}
\usepackage{graphicx}
\usepackage{subfig}

\XeTeXlinebreaklocale "zh"                      % 针对中文进行断行
\XeTeXlinebreakskip = 0pt plus 1pt minus 0.1pt  % 给予TeX断行一定自由度

\usefonttheme[onlylarge]{structurebold}
\setbeamerfont*{frametitle}{size=\normalsize,series=\bfseries}
\setbeamertemplate{navigation symbols}{}

% 修改图片显示设置
\setbeamertemplate{caption}[numbered]
\setbeamerfont{caption}{size=\footnotesize}
\setbeamerfont{captionname}{size=\footnotesize}

% 添加页码
\newcommand*\oldmacro{}
\let\oldmacro\insertshorttitle
\renewcommand*\insertshorttitle{
\oldmacro \hfill  \leftskip=.3cm
\insertframenumber\,/\,\inserttotalframenumber}

\title[2013年硕士毕业论文答辩:室内光无线通信技术研究与硬件设计]
{
\huge{室内光无线通信技术研究与硬件设计}
}
\normalsize
\author[潘乐园~100645\quad{}\quad{}\quad{}导师:赵春明~教授]
{
潘乐园\inst{*}~100645\\
~\\
导师:赵春明\inst{*}~教授
}

\institute[东南大学~移动通信国家重点实验室]
{
\inst{*}
东南大学~~移动通信国家重点实验室\and National Mobile Communications Research Laboratory\and Southeast University\and Nanjing, China, 210096
\and
\vskip-2mm
}

\date[WABI 2010]
{
2013年1月17日
}

% The main document

\begin{document}

% renew command
\renewcommand\contentsname{目录}
\renewcommand\listfigurename{插图目录}
\renewcommand\listtablename{表格目录}
\renewcommand\refname{参考文献}
\renewcommand\indexname{索引}
\renewcommand\figurename{图}
\renewcommand\tablename{表}
\renewcommand\abstractname{摘要}
\renewcommand\partname{部分}
\renewcommand\appendixname{附录}
\renewcommand\today{\number\year年\number\month月\number\day日}

\begin{frame}
	\titlepage
\end{frame}

\begin{frame}{目录}
	\tableofcontents
\end{frame}

\section{研究背景}
\subsection{室内光无线通信系统概况}
\begin{frame}{关于光无线通信}
近年来,随着无线通信技术的高速发展,射频频谱资源越来越紧张,利用现有射频频谱资源进行高速无线通信的实现成本已经越来越高。
电磁波辐射对人体带来的影响也逐渐受到人们的关注。
基于发光二极管(LED)的光无线通信系统因“绿色”、“安全”而成为研究热点。
\pause
\begin{block}{光无线通信系统与射频无线通信系统的对比}
\begin{itemize}
    \item 光:\alert{带宽不受限};射频:带宽受限(需考虑对其它用户的干扰)
    \item 光:\alert{保密性好};射频:保密性差
    \item 光:\alert{无需牌照};射频:需要牌照
    \item 光:\alert{“绿色”、“健康”};射频:电磁辐射对人体影响大
    \item 光:\textcolor{blue}{使用能量表达信号};射频:使用幅度和相位表达信号
    \item 光:\textcolor{blue}{路径损耗大};射频:相对于光,路径损耗小
    \item 光:\textcolor{blue}{噪声源包括热噪声和光噪声};射频:噪声源主要为热噪声
\end{itemize}
\end{block}
\end{frame}

\begin{frame}{光无线通信的研究成果简介}
\begin{itemize}
\item 2008年,Minh等人使用白色LED实现\alert{40Mbps}的通信速率,基带调制方案为OOK[Minh 2008]
\item 2009年,Minh等人使用白色LED实现\alert{100Mbps}的通信速率,基带调制方案为OOK[Minh 2009]
\item 2010年,Vucic等人使用白色LED作为发射源,Si-APD作为接收二极管,在使用蓝光滤光片的基础上实现了\alert{230Mbps}的通信速率,基带调制方案为OOK[Vucic 2010]
\item 2010年,Vucic等人在相同的硬件基础上,使用DCO-OFDM调制方案实现了\alert{513Mbps}的通信速率[Vucic 2010]
\item 2011年,Vucic等人将磷激发LED更换为RGB三色混光LED,在RGB三个通道中实现了\alert{803Mbps}的总通信速率[Vucic 2011]
\item 2012年,Kottke等人仅使用RGB三色混光LED中的红色通道实现了\alert{806Mbps}的通信速率,基带调制方案为DCO-OFDM[Kottke 2012]
\end{itemize}
\end{frame}

\begin{frame}{光无线通信系统的主要特性}
    \vspace{-0.5em}
    \begin{block}{信道特性}
        \begin{itemize}
        \item 信号能量衰减正比于距离的\alert{四次方}[Kahn 1997]
        \item 信道环境一般为\alert{单径LOS低通信道}
        \item 在不含直达径的环境中通信效率很低
        \end{itemize}
    \end{block}
    \pause
    \vspace{-0.5em}
    \begin{block}{光电器件}
        \begin{itemize}
        \item 电光器件:\alert{发光二极管(LED)}、激光二极管(LD)
        \item 光电器件:P-I-N光电二极管(PIN-PD)、雪崩光电二极管(APD)
        \end{itemize}
    \end{block}
    \pause
    \vspace{-0.5em}
    \begin{block}{光系统噪声}
        \begin{itemize}
        \item 电子元器件的\alert{热噪声}
        \item 光电二极管的\alert{暗电流噪声}
        \item \alert{背景光噪声},包括日光和照明用灯光
        \end{itemize}
    \end{block}
\end{frame}

\section{光无线信道建模}
\subsection{线性光无线信道建模}
\begin{frame}{线性光无线信道建模基础}
\begin{block}{线性光无线信道的建模基础与方法}
\begin{itemize}
\item   光无线通信系统主要受光传播信道和光电器件的影响:其中光传播信道为线性信道,而光电器件则为非线性系统。
        \alert{当发射功率较小时,光电器件工作于线性区间,则光系统表现为线性系统,可以在这个条件下建立线性光信道模型[Hranilovic 1999]}。
\item   建立线性光信道模型的基础是测量光信道时域冲激响应。
\item   \alert{论文中采用PN序列作为导频测量信道时域冲激响应[Gallager 2002]}。
\end{itemize}
\end{block}
\end{frame}
\begin{frame}{测量光信道时域冲激响应的理论基础}
导频信号为PN序列$u_j=\pm a$,满足
\begin{equation}
    \sum_{j=1}^n u_j u^*_{j+m} \approx
    \begin{cases}
    	na^2,  & m=0 \\
    	0,     & m\neq 0
    \end{cases}
    \quad{}\quad{}
    = a^2 n \delta_m
\end{equation}

接收信号为
\begin{equation}
	V_m = \sum_{k=0}^{n-1}h_{k,m}u_{m-k} + Z_m
\end{equation}

将导频信号与接收信号作相关操作,可得
\begin{equation}
	K_j = \sum_mV_mu^*_{m+j} = na^2h_{-j} + \sum_m Z_mu^*_{m+j}
\end{equation}

因此,信道的抽头系数为
\begin{equation}
	h_{-j} = \frac{K_{j}}{na^2} + Z'_j
\end{equation}
\end{frame}

\begin{frame}{使用PN序列测量信道时域冲激响应的误差分析}
使用PN序列作为导频测量信道冲激响应的测量误差为
\begin{equation}
	Z'_j = \frac{1}{na^2}\sum_{m=-j+1}^{-j+n}Z_mu^*_{m+j}
\end{equation}
因为$Z_m \sim N(0, N_0/2)$,其中,$N(\cdot, \cdot)$表示高斯分布,$N_0$表示噪声的功率谱密度,所以,
\begin{equation}
    \begin{split}
    	& E\{Z'_j\} = 0 \\
    	& VAR\{Z'_j\} = \frac{N_0}{2na^2}
    \end{split}
\end{equation}
其中,$E\{\cdot\}$表示统计期望,$VAR\{\cdot\}$表示统计方差。

从中可以看出,增大$n$的值,即使用长PN序列,可以有效的降低测量误差。
\end{frame}

\begin{frame}{测量与建模结果(1)}
    \begin{figure}[htbp]
        \centering
    	\includegraphics[width=0.8\textwidth]{figures/chapter-3/PN-TX-RX-250M.png}
    	\caption{发送的PN序列和接收到的PN序列}
    \end{figure}
\end{frame}

\begin{frame}{测量与建模结果(2)}
    \begin{figure}[htbp]
        \centering
    	\includegraphics[height=0.8\textheight]{figures/chapter-3/ChannelImpulseResponse.eps}
    	\caption{光信道时域冲激响应测量结果}
    \end{figure}
\end{frame}

\begin{frame}{测量与建模结果(3)}
    \begin{figure}[htbp]
        \centering
        \subfloat[幅频响应]{
            \includegraphics[width=0.5\textwidth]{figures/chapter-3/ChannelImpulseResponse-AmpFreqResponse.eps}
        }
        \subfloat[相频响应]{
            \includegraphics[width=0.5\textwidth]{figures/chapter-3/ChannelImpulseResponse-PhsFreqResponse.eps}
        }
        \caption{光信道时域冲激响应所对应的频域特性}
    \end{figure}
\end{frame}

\subsection{非线性光无线信道建模}
\begin{frame}{非线性光无线信道的建模基础}
\begin{block}{非线性光无线信道的建模基础与方法}
\begin{itemize}
\item   光无线通信信道受发光二极管和接收二极管的影响。
        \alert{在大功率发射情景下,发光二极管和接收二极管均会出现非线性失真。
        为了更深入的研究光无线通信信道,需要建立更加符合实际情况的非线性光信道模型。}
\item   分析模型与\alert{经验模型}[Gharaibeh 2012]。
\item   论文中采用\alert{经验模型(Three-Box模型)}建立了光无线通信系统的非线性信道模型。
\end{itemize}
\end{block}
\end{frame}

\begin{frame}{非线性光无线信道的建模方法}
    \begin{figure}[htbp]
        \centering
    	\includegraphics[width=0.7\textwidth]{figures/chapter-3/ThreeBoxModel.eps}
    	\caption{Three-Box模型框图}
    \end{figure}
    论文中所采用的经验模型为Three-Box模型,此模型包括三个部分:输入滤波器、输出滤波器和无记忆非线性失真模块。
    
    \pause
    输入输出滤波器的传递函数分别为
    \begin{equation}
        H_1(f) = \frac{H_{ss}(f)}{G_{ss}\vert H_{sat}(f)\vert}
    \end{equation}
    和
    \begin{equation}
        H_2(f) = \frac{H_{sat}(f)}{\vert G_{sat}\vert}
    \end{equation}
\end{frame}

\begin{frame}{非线性光无线信道的建模方法(续)}
    无记忆非线性失真模块的多项式表达式为
    \begin{equation}
    	y_n = \sum_{k=1}^N b_k\vert x_n\vert^{k-1}x_n
    \end{equation}
    使用矩阵形式表示为
    \begin{equation}
    	\mathbf{Y} = \boldsymbol{\Phi}_\mathbf{X} \mathbf{b}
    \end{equation}
    使用最小二乘(Least squares, LS)算法拟合出的多项式系数为
    \begin{equation}
    	\alert{\mathbf{b} = \Phi_\mathbf{X}^\dag \mathbf{Y}}
    \end{equation}
    其中,$\Phi_\mathbf{X}^\dag = (\Phi_\mathbf{X}^\mathbf{H}\Phi_\mathbf{X})^\mathbf{-1}\Phi_\mathbf{X}^\mathbf{H}$表示$\Phi_\mathbf{X}$的Moore-Penrose广义逆。
\end{frame}

\begin{frame}{测量与建模结果(1)}
    测得的小信号(Output Power: -20dBm)频率响应结果:
    \begin{figure}[htbp]
        \centering
        \subfloat[幅频响应]{
            \includegraphics[width=0.5\textwidth]{figures/chapter-3/SmallSignalAmpFreqResponse.eps}
        }
        \subfloat[相频响应]{
            \includegraphics[width=0.5\textwidth]{figures/chapter-3/SmallSignalPhsFreqResponse.eps}
        }
        \caption{系统小信号频率响应测量值}
    \end{figure}
\end{frame}

\begin{frame}{测量与建模结果(2)}
    测得的大信号(Output Power: +5dBm)频率响应结果:
    \begin{figure}[htbp]
        \centering
        \subfloat[幅频响应]{
            \includegraphics[width=0.5\textwidth]{figures/chapter-3/LargeSignalAmpFreqResponse.eps}
        }
        \subfloat[相频响应]{
            \includegraphics[width=0.5\textwidth]{figures/chapter-3/LargeSignalPhsFreqResponse.eps}
        }
        \caption{系统大信号频率响应测量值}
    \end{figure}
\end{frame}

\begin{frame}{测量与建模结果(3)}
    测得的功率(Reference Frequency: 10MHz)响应结果:
    \begin{figure}[htbp]
        \centering
        \subfloat[AM-AM参数]{
            \includegraphics[width=0.5\textwidth]{figures/chapter-3/AmpPowerResponse.eps}
        }
        \subfloat[AM-PM参数]{
            \includegraphics[width=0.5\textwidth]{figures/chapter-3/PhsPowerResponse.eps}
        }
        \caption{系统功率响应测量值}
    \end{figure}
\end{frame}

\begin{frame}{测量与建模结果(4)}
    输入输出滤波器的幅频响应与相频响应建模结果:
    \begin{figure}[htbp]
        \centering
        \subfloat[幅频响应]{
            \includegraphics[width=0.5\textwidth]{figures/chapter-3/ThreeBox-H1H2-gain.eps}
        }
        \subfloat[响应频响应]{
            \includegraphics[width=0.5\textwidth]{figures/chapter-3/ThreeBox-H1H2-angle.eps}
        }
        \caption{Three-Box模型输入滤波器$H_1$和输出滤波器$H_2$的频率响应曲线}
    \end{figure}
\end{frame}

\begin{frame}{测量与建模结果(5)}
    使用5阶失真多项式拟合出的非线性失真模块多项式系数为:
    \begin{table}[htbp]
        \caption{根据实测参数拟合出的AM-AM和AM-PM失真模块多项式系数}
        \centering
        \begin{tabular}{cc}
            \toprule
            多项式系数名称 & 多项式系数值\\
            \midrule
            $b_1$ & $-0.1273 + 1.0322i$\\
            $b_3$ & $0.1599 - 0.4940i$\\
            $b_5$ & $-0.0800 + 0.0695i$\\
            \bottomrule
        \end{tabular}
    \end{table}
\end{frame}

\begin{frame}{测量与建模结果(6)}
    实测结果与模型输出结果对比:
    \begin{figure}[htbp]
        \centering
        \subfloat[幅度响应曲线]{
            \includegraphics[width=0.5\textwidth]{figures/chapter-3/ThreeBox-AMAM.eps}
        }
        \subfloat[相位响应曲线]{
            \includegraphics[width=0.5\textwidth]{figures/chapter-3/ThreeBox-AMPM.eps}
        }
        \caption{Three-Box模型信道响应与实测值对比图}
    \end{figure}
\end{frame}

\begin{frame}{测量与建模结果(7)}
    建模误差(相对于实测数据):
    \begin{figure}[htbp]
        \centering
    	\includegraphics[height=0.7\textheight]{figures/chapter-3/ThreeBox-MSE.eps}
    	\caption{Three-Box模型建立的信道模型的均方误差值(MSE)}
    	\label{fig:ThreeBoxVSMeasurementMSE}
    \end{figure}
\end{frame}

\section{光无线通信中的基带技术研究}
\subsection{光无线通信中基带调制技术的研究}
\begin{frame}{光无线通信中的基带调制技术}
    光无线通信系统是基带直流偏置系统,需要对传统的基带调制方案加以改进。
    光系统中各种基带调制方案的\alert{带宽效率}和\alert{功率效率}对比如下:
    \begin{itemize}
    \item 带宽效率:M-DCO-PAM > NRZ-OOK > RZ-OOK > M-PPM
    \item 功率效率:M-DCO-PAM > M-PPM > RZ-OOK > NRZ-OOK
    \item M-PPM的带宽效率随着调制阶数M的提高而下降,但功率效率随着调制阶数的提高而上升
    \item M-DCO-PAM的带宽效率均随着调制阶数M的提高而上升,但是随着M的提高,需要更高的信噪比才能保证达到相同的误比特率
    \end{itemize}
    \alert{在低通系统中,需要很复杂的均衡器才能实现宽带单载波通信,而OFDM技术可以有效地解决这个问题。}
    
    \alert{因此,论文基于DCO-OFDM研究了自适应调制技术在光无线通信系统中的应用。}
\end{frame}

\begin{frame}{自适应调制在光无线通信中的应用}
    自适应调制(Adaptive modulation, AM)技术可以有效地应对不同的接收信噪比环境,是保证服务质量(QoS)的有效技术。
    尤其是OFDM调制,可以在不同子载波上分别采用合适的调制方案,自由度更高[Xu 2005]。
    OFDM中自适应调制的根本原理是,对于每个子载波,根据所估计的等效信噪比选择适当的调制方案,从而在保证整体误比特率的前提下,提高系统频谱效率。
    \pause
    \begin{block}{自适应调制算法}
    \begin{itemize}
    \item 所选择的目标误比特率为$\text{BER}=2\times10^{-3}$[Vucic 2011]。
    \item 可选调制方案集合为
    $
        \{\text{\small{不传输, BPSK, QPSK, 8-QAM, 16-QAM, 32-QAM, 64-QAM, 128-QAM, 256-QAM}}\}
    $
    \item 每种调制方案对应的等效信噪比门限为
    $
        \{0, \tilde\gamma_{BPSK}, \tilde\gamma_{QPSK}, \tilde\gamma_{8-QAM}, \tilde\gamma_{16-QAM}, \tilde\gamma_{32-QAM}, \tilde\gamma_{64-QAM}, \tilde\gamma_{128-QAM}, \tilde\gamma_{256-QAM}\}
    $
    \item 对于每个子载波,根据计算所得的等效信噪比选择并使用频谱效率尽量高的调制方案。
    \end{itemize}
    \end{block}
\end{frame}

\begin{frame}{仿真结果(1)}
    \begin{figure}[htbp]
        \centering
        \subfloat[固定调制对应的误比特率性能曲线]{
            \includegraphics[width=0.5\textwidth]{figures/chapter-4/FM-BER.eps}
        }
        \subfloat[自适应调制对应的误比特率性能曲线]{
            \includegraphics[width=0.5\textwidth]{figures/chapter-4/AM-BER.eps}
        }
        \caption{光无线信道下固定调制与自适应调制对应的误比特率曲线对比图}
    \end{figure}
\end{frame}

\begin{frame}{仿真结果(2)}
    \begin{figure}[htbp]
        \centering
    	\includegraphics[height=0.7\textheight]{figures/chapter-4/AM-VS-FM-SpectralEfficiency.eps}
    	\caption{光无线信道下自适应调制(AM)和固定调制(FM)的频谱效率对比图}
    \end{figure}
\end{frame}

\subsection{均衡技术在光无线通信中的应用}
\begin{frame}{数字预均衡}
在室内光无线通信系统中,可以在估计信道参数之后,在发射端做预均衡,从而有效扩展通频带。
这里,将不考虑信道估计和信道反馈过程,认为信道估计是理想的,通过推导设计出满足最小二乘(LS)准则的数字预均衡滤波器。
\pause
\begin{block}{使用LS准则估计预均衡滤波器系数}
接收信号为
\begin{equation}
    y(n) = x(n) * g(n) * h(n) + z(n)
\end{equation}
改写成矩阵形式可得
\begin{equation}
    \mathbf{y} = \mathbf{XHg} + \mathbf{z}
\end{equation}
使用LS准则估计预均衡滤波器$\mathbf{g}$可得
\begin{equation}
    \alert{\hat{\mathbf{g}} = \arg\underset{\mathbf{g}}{\min}\{\|\mathbf{XHg - X}\|^2\} = \mathbf{H^\dag D}}
\end{equation}
其中,$\mathbf{H^\dag = (H^HH)^{-1}H^H}$表示信道矩阵$\mathbf{H}$的Moore-Penrose广义逆,$\mathbf{D}=[1,0,\cdots,0]^T$为单位冲激向量。
\end{block}
\end{frame}

\begin{frame}{数字预均衡的仿真结果(1)}
    \begin{figure}[htbp]
        \centering
    	\includegraphics[height=0.7\textheight]{figures/chapter-4/Pre-Equalizated-Channel.eps}
    	\caption{预均衡前后信道幅频响应对比图}
    \end{figure}
\end{frame}

\begin{frame}{数字预均衡的仿真结果(2)}
    \begin{figure}[htbp]
        \centering
        \subfloat[发送信号]{
            \includegraphics[width=0.5\textwidth]{figures/chapter-4/Pre-Equalization-Transmitting-Signal.eps}
        }
        \subfloat[接收信号]{
            \includegraphics[width=0.5\textwidth]{figures/chapter-4/Pre-Equalization-Received-Signal.eps}
        }
        \caption{100Mbps OOK 调制预均衡与不预均衡的接收信号对比图}
    \end{figure}
\end{frame}

\begin{frame}{数字预均衡的仿真结果(3)}
    \begin{figure}[htbp]
    \centering
	\includegraphics[height=0.7\textheight]{figures/chapter-4/BER-PreEqualization.eps}
	\caption{预均衡前后的误比特率对比图}
\end{figure}
\end{frame}

\begin{frame}{模拟后均衡}
    模拟后均衡电路可以有效降低发射机和接收机的基带计算复杂度。
    虽然无法获得的均衡效果比数字预均衡器差一些,但仍然可以有效地扩展通频带,同时不改变基带计算复杂度。
    下图给出了一阶模拟后均衡电路结构。
    \begin{figure}[htbp]
        \centering
    	\includegraphics[height=0.4\textheight]{figures/chapter-4/AnalogPostEqualizationCircuits.eps}
    	\caption{接收前端模拟后均衡电路结构图}
    \end{figure}
\end{frame}

\begin{frame}{模拟后均衡电路实测结果}
    \begin{figure}[htbp]
    \centering
    \subfloat[均衡前后信道幅频响应对比]{
        \includegraphics[width=0.5\textwidth]{figures/chapter-4/EqualizedVSNonEqualizedAmpFreqResponse.eps}
    }
    \subfloat[均衡前后信道相频响应对比]{
        \includegraphics[width=0.5\textwidth]{figures/chapter-4/EqualizedVSNonEqualizedPhsFreqResponse.eps}
    }
    \caption{使用模拟均衡器的信道频响与不适用模拟均衡器的信道频响对比图(\alert{$R = 150\Omega$,$C = 100pF$})}
\end{figure}
\end{frame}

\section{光无线通信硬件设计}
\subsection{光无线通信硬件设计原理}
\begin{frame}{硬件设计结构}
    \begin{figure}[htbp]
        \centering
    	\includegraphics[width=0.8\textwidth]{figures/chapter-5/VideoTXRX.eps}
    	\caption{演示平台结构简图}
    \end{figure}
\end{frame}

\begin{frame}{发射机设计}
    \vspace{-0.5em}
    \begin{figure}[htbp]
        \centering
    	\includegraphics[height=0.4\textheight]{figures/chapter-5/TransmitterDetails.eps}
    	\caption{室内光无线通信系统发射机设计方案详细结构图}
    \end{figure}
    \vspace{-1.5em}
    \begin{figure}[htbp]
        \centering
    	\includegraphics[height=0.35\textheight]{figures/chapter-5/TransmitterPhoto.png}
    	\caption{室内光无线通信系统发射机实物图}
    \end{figure}
\end{frame}

\begin{frame}{接收机设计}
    \vspace{-0.5em}
    \begin{figure}[htbp]
        \centering
    	\includegraphics[height=0.4\textheight]{figures/chapter-5/ReceiverDetails.eps}
    	\caption{室内光无线通信系统接收机设计方案详细结构图}
    \end{figure}
    \vspace{-1.5em}
    \begin{figure}[htbp]
        \centering
    	\includegraphics[height=0.35\textheight]{figures/chapter-5/ReceiverPhoto.png}
    	\caption{室内光无线通信系统接收机实物图}
    \end{figure}
\end{frame}

\subsection{光无线通信演示效果}
\begin{frame}{演示平台效果图}
    \begin{figure}[htbp]
        \centering
    	\includegraphics[height=0.8\textheight]{figures/chapter-5/VideoTransmission.png}
    	\caption{室内光无线通信系统视频传输演示平台}
    \end{figure}
\end{frame}

\section{总结}
\begin{frame}{总结}
    论文主要研究内容包括:
    \begin{itemize}
        \item 详细介绍了光无线通信系统的结构与特性
        \item 使用实测数据建立了光无线通信系统的线性和非线性\alert{信道模型}
        \item 在所建立的信道模型基础之上深入研究了光无线通信中的各种\alert{基带技术},包括基带高效调制技术、自适应调制技术和均衡技术
        \item 完成了\alert{硬件验证平台}的设计与调测,验证了光无线通信系统的可实现性
    \end{itemize}
\end{frame}

\section*{致谢}
\begin{frame}
	\begin{center}
		\Huge{\bf{谢\quad{}谢~!}}\\
        \vspace{1em}
        \Huge{Q\textcolor{gray}{uestion} and A\textcolor{gray}{nswer}}
	\end{center}
\end{frame}


\end{document}
