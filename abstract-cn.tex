% !Mode:: "TeX:UTF-8"

% 中文摘要
\begin{abstract}{可见光通信\quad{}组网\quad{}最优化\quad{}灯组调度\quad{}增量式调度}

近年来,无线通信的发展受到了诸多的限制,如频谱资源的匮缺,射频对身体的辐射危害等。而随着LED技术的不断发展和在现代家庭中的不断应用,与之对应的使用LED灯进行数据传输的室内可见光通信技术受到了极大的关注,并被进行了大量的深入研究。可见光通信技术具有频谱资源丰富,绿色安全,能量转化率高等诸多优点,因此对于实现室内高速绿色通信具有重要的研究价值。本文主要着眼于室内可见光通信系统的组网技术,以解决典型室内灯组组网下的多用户的高效通信问题。

首先,本文介绍了可见光通信及其组网技术的研究背景和研究进展。自提出可见光通信的概念起,可见光通信的研究就在全球各地被广泛开展,但是目前研究较多的仍然是可见光通信的物理层的实现,以获得更高的信息传输速率作为目标。同时,本文也介绍了可见光通信在组网技术方面的一些理论研究成果。另外,本文还对可见光通信系统中的基础理论进行了相关介绍,如光通信的链路方式,信道分析等内容,并阐述了可见光通信在组网方面可能会遇到的一些技术问题。

其次,本文对可见光通信系统中的LED灯组的最优化布局问题进行了研究。对于该问题,本文并不是单纯地考虑覆盖的均匀度,而是以光照强度和光接收功率的均匀度作为一个目标,以接收平面的平均光接收功率值作为另一个目标,以其他条件作为约束,建立起了一个多目标的最优化模型,并将该多目标最优化模型转变为单目标最优化模型进行了求解。仿真结果表明,本模型可以通过合理地配置模型权重系数,调节对于灯组布局的不同需求,并求解出合理的最优化灯组布局结果。

接着,本文研究了分布式灯组架构下的灯组调度问题。分布式灯组是指所有的灯组都连接下同一个调度器上,受该调度器统一控制。在该架构下,本文提出了一种基于用户运动方向的灯组协同调度算法,该算法在灯组和用户进行数据通信时,测量用户的上行功率,利用一段时间内接收到的功率信息,获得用户的运动方向信息,并利用该用户的运动方向信息和当前测量到的功率信息,进行灯组的调度,从而保证用户在室内移动时的可靠通信。仿真结果表明,该灯组调度方法在数据重传方面的表现显著优于不使用灯组协作的调度算法和简单的灯组协同调度算法,可以满足系统对可靠通信的要求。

最后,本文还研究了集中式的灯组架构下的灯组调度问题。集中式的灯组则意味着各灯组间可以独立地发送数据,相互之间不影响。同样在该架构下,本文提出了一种增量式的灯组调度算法。该增量式的灯组调度算法可以分为两个阶段,分别是长周期的全局调度阶段和短周期的局部调度阶段。其中全局调度是针对所有的用户,以整体的系统容量为目标,将所有的用户划分在多个时隙中,确保每个时隙中的用户之间没有干扰。而低复杂度的局部调度则是跟踪移动用户的变化,并调整之前产生的灯组调度结果以适应用户的移动。仿真结果表明,该增量式灯组调度方法对于多用户运动的场景,可以在获得高系统容量的同时,可以显著降低系统的调度复杂度,完成系统的灯组调度。

\end{abstract} 