% !Mode:: "TeX:UTF-8"
% 全文总结与展望

\chapter{全文总结与展望}\label{chap:conclusion}
随着LED技术在室内照明领域的大规模使用,使用LED灯组进行通信的室内可见光通信技术也悄然兴起。在室内可见光通信系统中,LED灯组既起到照明的作用,又担负着数据传输的功能。自2000年开始,各国的研究学者已经对可见光通信技术的底层技术进行了较为详细的分析,尤其是对于点对点传输,目前最高速率已经可以达到803Mb/s。
本文则在基础上,对上层的光通信组网技术进行了研究,研究内容包括灯组的最优分布,灯组的调度策略等领域,主要目标是形成一个高效可靠的调度方案,来解决多移动用户下的可见光通信的调度问题。
\section{论文全文总结}
本文第一部分首先介绍了可见光通信的研究背景和研究意义,介绍了本课题开展的时代背景和对社会的重要意义,
并介绍了可见光通信的发展历程和国内外科研工作者关于该领域的研究进展,同时也重点介绍了在本文所研究的组网技术方面目前国内外研究的主要趋势和成果。

本文的第二部分主要是对可见光通信系统的基本理论知识进行了说明,为之后的章节的理论分析奠定了基础。主要介绍了可见光通信和常用的室内光通信的区别,光通信的链接方式,信道分析等内容,从而对可见光通信的基本结构有一定的了解。
同时在这部分中还介绍了一般系统的组网方式和在可见光通信组网方面可能会使用到的关键技术,从而为下文的研究指明方向。

本文的第三部分对系统的灯组的最优化布局进行了分析。在室内可见光通信系统中,灯组的最优化布局可以均匀化用户接收平面的光照分布和光接收功率分布,从而为用户的可靠通信打下基础。
本文提出了一种多目标规划模型,来提高光照强度和光接收功率的均匀度和用户平面的平均接收功率。
在求改该多目标规划模型后,本文给出了三组仿真结果,分别表示偏重于均与度的最优灯组布局效果,偏重于平均光接收功率的最优灯组布局效果和综合考虑上述两个方面的最优灯组布局效果。

本文的第四部分,则是对分布式灯组结构下的可见光通信系统的灯组调度进行了研究。
该结构下的所有灯组统一由调度器控制进行数据传输。本文在分析了该灯组架构的系统的特征之后,提出了一种基于用户方向的灯组协同调度技术。
在进行调度的过程中,不仅依据测量的在数据通信中用户上行的功率,同时利用当前用户的移动方向信息,从而调度最适合的灯组集合为用户提供服务。
这种调度算法,由于在调度过程中增加了用户的方向信息,可以提供更加准确的灯组集协同工作,提供下行数据传输,从而解决了用户在灯组边缘区误码率过高这个问题。
本部分还提供了多组的仿真结果,已验证所提出算法的优越性。

本文的第五部分,则是对独立式灯组结构下的可见光通信系统的灯组调度进行了分析。
在该架构下,灯组可以独立发送数据信息,从而可以空分复用为用户提供服务,提高系统容量值。
对于该灯组架构,本文也提出了一种增量式的灯组调度算法,在解决用户间干扰的情况下,尽可能提高系统的容量。
该增量式调度算法,主要分为两个阶段,长周期的全局调度和短周期的局部调度。其中,全局调度用于产生无用户间干扰和高系统容量的调度结果,
而局部调度算法则用于不断调整原有的调度结果,以适应移动用户的位置改变。同样,本部分也给出了一系列的仿真结果图,已验证算法的可靠度和准确性。

第六章总结全文工作,并给出了室内可见光通信系统组网技术的进一步研究方向。

\section{进一步的研究方向}
本文上述的研究只是对光通信组网中的系统级调度方面进行了基本的分析,但这只是组网技术中的一个基本问题,用于解决多用户的高效数据通信问题。
在本文的调度策略中,为了保证用户的通信质量,都采用了一个核心思路,即通过灯组的协作为用户提供服务。因此,对于组网技术的研究,
还可以从另外一个角度分析,将每个灯组看作类似蜂窝网中的一个基站,而室内的可见光系统可以看作是多个灯组基站组网的一个网络,进而可以研究以下一些问题:

\begin{enumerate}
    \item 在用户的移动过程中,如何进行多个灯组的切换操作,以保证通信连接的持续性,针对于在室内灯组基站分布比较密集的现状,当用户的移动速度较快时,如何设计快速切换来保证用户通信质量;
    \item 在每个灯组作为一个基站的系统中,如何正确处理好新的用户接入和原有用户数据通信之间的关系,是否需要使用资源预留的方式来防止新用户接入失败;
    \item 由于LED灯组是遍布于室内的基础设备,因此在可见光通信系统中,可以研究有效的多灯组的定位算法,在进行数据通信的同时,建立多用户室内定位系统,进而提供多种室内定位服务。
\end{enumerate}

随着对可见光通信系统的组网技术不断深入研究,光通信将会更加实用的形式展现在大众面前,同时随着LED技术的大规模普及,可以实现超高速率的可见光通信技术将会离人们的生活越来越近。
